% Template authors: Bogumił Kamiński and Michał Jakubczyk

\documentclass[12pt,a4paper,twoside,openany]{book}
\usepackage[T1]{fontenc}
\usepackage{amsfonts}
\usepackage{physics}
\usepackage[utf8]{inputenc}
\usepackage[polish]{babel}
\usepackage{graphicx}
\usepackage{times}
\usepackage{indentfirst}
\usepackage[left=3cm,right=2cm,top=2.5cm,bottom=2.5cm]{geometry}
\usepackage{natbib}
\usepackage{enumitem}
\usepackage{color}
\usepackage{soul}
\usepackage{tikz}
\usepackage{url}
\usepackage{todonotes}
\usepackage{verbatim}
\setlist{itemsep=0pt}
\setlist{nolistsep}
\frenchspacing
\linespread{1.3}
\addto\captionspolish{%
\renewcommand*\listtablename{Spis tabel}
\renewcommand*\tablename{Tabela}
}
\usepackage{titlesec}
\titlelabel{\thetitle.\quad}

\frenchspacing

\begin{document}

\begin{center}
\includegraphics[scale=0.3]{sgh_full.png}

\vspace{1cm}

% tu i dalej fbox należy usunąć i wpisać odpowiednią wartość
Studium magisterskie
\end{center}

\vspace{1cm}

\noindent Kierunek: Analiza danych - big data

\vspace{1cm}

{
\leftskip=10cm\noindent
Aleksander Wojnarowicz\newline
Nr albumu: 77438

}

\vspace{2cm}

\title{Wzór pracy dyplomowej SGH}
\makeatletter

\begin{center}
\LARGE\bf
\fbox{\@title}
\end{center}

\vspace{2cm}

{
\leftskip=10cm\noindent
Praca magisterska
napisana w\newline
Instytucie Ekonometrii\newline
pod kierunkiem naukowym\newline
dr Sebastiana Zająca

}

\vfill

\begin{center}
Warszawa, \the\year
\end{center}
\thispagestyle{empty}

\clearpage
\thispagestyle{empty}
\mbox{}
% druga strona będzie pusta, ponieważ drukujemy dwustronnie
% a mbox jest po to, żeby ta strona się pokazała
% od procenta robimy komentarze
\clearpage

\tableofcontents

\clearpage

\chapter*{Uwagi techniczne, skasuj rozdział po uwzględnieniu}

Wymagane jest wgranie pracy do systemu Overleaf (\url{https://www.overleaf.com/}) i udostępnienie promotorowi z~prawem zmian.

Pytania do promotora w~tekście proszę zadawać \todo[inline]{o tak}, tj.~stosując \verb!\todo[inline]{o tak}!.

Po kropkach niekończących zdania (np.~po `np.', `tj.', `itd.', itd.) stawiamy znak tyldy, tj.~\verb!~! (żeby zmniejszyć odstęp). Wykorzystujemy także tyldę między odwołaniem i~numerem, np.~`Tabela~3' (oczywiście odwołanie robione automatycznie przez \verb!\ref{}!). I wreszcie, stosujemy tyldę po pojedynczych literach (np.~`i', `a'), bo to niełamliwa spacja i~gwarantuje brak takich liter na końcu linii.

Przykłady odwołań źrółowych:
\begin{itemize}
\item \citep{dolan2000}, \citet{dolan2000}, \defcitealias{dolan2000}{Dolana (2000)}\citetalias{dolan2000};
\item \citep{wakker1999}, \citet{wakker1999}, \defcitealias{wakker1999}{Wakkera i Zhanka (2000)}\citetalias{wakker1999};
\item \citep{drummond1997}, \citet{drummond1997}, \defcitealias{drummond1997}{Drummonda et al.~(2000)}\citetalias{drummond1997}.
\end{itemize}

Co warto przeczytać o~\LaTeX:
\begin{enumerate}
\footnotesize % tak zmieniamy rozmiar fontu
\item \url{http://www.tex.ac.uk/ctan/info/gentle/gentle.pdf},
\item \url{ftp://sunsite.icm.edu.pl/pub/CTAN/info/lshort/english/lshort.pdf},
\item \url{http://paws.wcu.edu/tsfoguel/tikzpgfmanual.pdf},
\item \url{https://www.overleaf.com/latex/learn/free-online-introduction-to-latex-part-1}.
\end{enumerate}

\chapter{Wprowadzenie}

Nierelatywistyczna fizyka klasyczna stanowi podstawę obrazu świata, postrzeganego oczami ludzkiej wyobraźni. Umysł homo sapiens wyewoluował w otoczeniu obiektów makroskopowych, a ponadto nigdy nie doświadczył (ani nawet nie był w stanie sobie wyobrazić) prędkości choć trochę zbliżonych do prędkości światła. Z tychże powodów, dwa wielkie osiągnięcia współczesnej fizyki – mechanika kwantowa oraz teoria względności (zarówno ogólna, jak i szczególna) – są tak nieintuicyjne. Tymczasem płynące z nich wnioski i wyniki stanowią podstawę dzisiejszej cywilizacji. Dylatacja czasu jest istotnym czynnikiem, jaki muszą uwzględniać precyzyjne systemy nawigacyjne GPS. Tranzystor z kolei – podstawowy budulec układów scalonych, a zatem również wszechobecnej dziś elektroniki – powstał i był doskonalony dzięki badaniom nad kwantowymi własnościami materii w nanoskali.

\section{Cel pracy}
Celem niniejszej pracy jest zbadanie oraz przeanalizowanie pewnego interesującego zastosowania fizyki mikroświata w dziedzinie finansów, a konkretnie - budowie portfela inwestycyjnego opartego o indeks giełdowy S\&P 500. Poza oceną samych rezultatów opisanej w kolejnych rozdziałach procedury istotne jest też porównanie ich do wyników jakie można otrzymać, stosując rozwiązania klasyczne tj. zaimplementowane na komputerze niekwantowym, który ponadto nie symuluje komputera kwantowego. Porównanie to dotyczy nie tylko wydajności procedur, czy też tego, jakie komponenty indeksu zostaną wskazane, ale także ocenioną a posteriori stopą zwrotu z wygenerowanych portfeli.

\chapter{Współczesna teoria portfela inwestycyjnego}

\chapter{Dywersyfikacja portfolio w ujęciu teorii grafów}
Zaprezentowane w poniższym rozdziale sposoby patrzenia na, oraz rozwiązywania zadania dywersyfikacji portfolio nie są jedynymi możliwymi. Stanowią raczej interesującą alternatywę dla tradycyjnych metod doboru spółek do portfela, jakimi są m.in. \textit{Tutaj wymieniam kilka popularnych, niegrafowych sposobów na dywersyfikację portfolio, wraz z przypisami}

\section{Wprowadzenie do teorii grafów}

\section{Dywersyfikacja jako problem \textit{MIS}}

\section{Klasyczne rozwiązania problemu \textit{MIS}}

\chapter{Mechanika kwantowa i algorytm \textit{Quantum Annealing}}

Quantum Computing (QC) to stosunkowo młody obszar badań naukowych, zajmujący się wykorzystaniem maszyn cyfrowych, które przetwarzają informacje używając tzw. \textit{qubitów}. Aby zrozumieć potencjał tych urządzeń dla współczesnej informatyki, niezbędne jest wprowadzenie pojęć \textit{superpozycji} oraz \textit{splątania kwantowego} - tematy zostaną poruszone w dalszej części rozdziału. Ogólnie jednak przewaga komputerów kwantowych zasadza się w głównej mierze na dwóch aspektach. Pierwszy z nich, polega na tym, że zjawiska kwantowe pozwalają takiemu komputerowi przeprowadzać wiele różnych obliczeń \textit{jednocześnie} na tej samej porcji pamięci (nie jest to więc to samo, co \textit{parallel computing}, czy nawet \textit{multithreading}). Druga zaś kwestia dotyczy szerokiej klasy problemów, które nie mogą być rozwiązane na komputerze klasycznym (lub przynajmniej nie w rozsądnym przedziale czasu), za to posiadają efektywną i naturalną implementacje na komputerach kwantowych\footnote{Przykładem takiego problemu jest m.in symulacja systemów molekularnych.}. Niniejsza praca skupiać się będzie tylko i wyłącznie na zbadaniu pierwszego z wymienionych aspektów.

W następnej sekcji poruszony zostanie także temat ograniczeń, jakie prawa fizyki nakładają na klasyczne maszyny cyfrowe, a w szczególności na skalowalność ich mocy obliczeniowej. Istnienie tych restrykcji dostarcza dodatkowych powodów, dla których badania w obszarze kwantowej informatyki stanowić będą jeden z kluczowych zagadnień w rozwoju nowoczesnej technologii.



\section{Granice klasycznej informatyki}

\subsection{Tranzystory i elektronika}
Podstawą działania nie tylko komputerów, ale w zasadzie niemal całej współczesnej elektroniki są tranzystory – mikroskopijne „przełączniki”, które, w zależności od stanu w jakim się znajdują, mogą albo przepuszczać, albo zatrzymywać elektrony płynące przez obwody elektryczne. Te dwa stany koduje się jako 0 (przełącznik nie przepuszczający elektronów) oraz 1 (przełącznik przepuszczający elektrony). Takie podejście stanowi bazę kodu binarnego, na którym oparte jest działanie komputerów.
Na bazie tranzystorów konstruowane są bramki logiczne, które z kolei pozwalają na przeprowadzanie podstawowych operacji arytmetycznych, wykonywanych na wyżej wspomnianym kodzie binarnym. Poprzez składanie ze sobą wielu różnych tego typu operacji otrzymujemy pożądane zachowania maszyn elektronicznych – obliczenia naukowe, grafikę komputerową, komunikację internetową etc.
	Celem zwiększenia mocy obliczeniowych maszyny elektronicznej, konieczne jest zwiększenie liczby tranzystorów, które zawiera w sobie ta maszyna. Istotne jest jednak, aby nie wpłynęło to zbytnio na rozmiar urządzenia – powinno być najmocniejsze jak to tylko możliwe, będąc przy tym najmniejsze jak to tylko możliwe. Z tego względu na przestrzeni lat następował konsekwentny spadek rozmiaru tranzystora (patrz. Tabela 1).
	


\begin{table}[h]
\centering
\caption{Tranzystory na przestrzeni lat}
\label{tab:przyklad}
\footnotesize
\begin{tabular}{|l|r|}
\hline
Rok & Rozmiar [nm]\\
\hline
1971 & 10000  \\
1974 & 6000  \\
1977 & 3000 \\
1981 & 1500 \\
1984 & 1000 \\
1987 & 800 \\
1990 & 600 \\
1993 & 350 \\
1996 & 250 \\
1999 & 180 \\
2001 & 130 \\
2003 & 90 \\
2005 & 65 \\
2007 & 45 \\
2009 & 32 \\
2012 & 22 \\
2014 & 14 \\
2016 & 10 \\
2018 & 7 \\
2020 & 5 \\

\hline
\end{tabular} 
\end{table}

\subsection{Tunelowanie kwantowe}

Ten postępujący wzrost mocy obliczeniowej ma jednak granicę, wyznaczoną przez prawa fizyki kwantowej. W miarę jak tranzystory kurczą się coraz bardziej do rozmiarów porównywalnych z rozmiarami atomów, coraz istotniejszą rolę gra zjawisko \textit{tunelowania kwantowego}. Z pewnym prawdopodobieństwem, które rośnie odwrotnie do rozmiarów „przełącznika”, elektron jest w stanie pokonać barierę potencjału, nawet przy braku odpowiedniej ilości energii, której wymaga do tego fizyka klasyczna. Innymi słowy, istnieje zawsze niezerowe prawdopodobieństwo, że pomimo iż tranzystor będzie w stanie 0, to elektrony i tak będą mogły się przez niego przedostać . Oczywistą konsekwencją tego zjawiska jest niepoprawne działanie elektroniki – wyniki pracy maszyn nabrałyby charakteru probabilistycznego, a zatem nie można by było na nich polegać.
	
Prawdopodobieństwo zajścia kwantowego tunelowania zadane jest, przy pewnych założeniach, przez wzór\footnote{V. Silva, \textit{Practical Quantum Computing For Developers}, Apress 2018} :
\begin{equation}
e^{-\frac{4a\pi}{h}\sqrt{2m(V - E)}}
\end{equation}
gdzie: \\
\indent a – grubość bariery (w naszym przypadku – rozmiar tranzystora) \\
\indent h – stała Planka ($ \approx 6.626 \cdots 10^{-34}  kg  \cdot m^2 / s$) \\
\indent m – masa cząstki (dla elektronu $m \approx 9,109 3837015 \cdot 10^{-31} kg$) \\
\indent V – energia potencjalna bariery \\ 
\indent E – energia kinetyczna cząsteczki ( dla zjawiska tunelowania kwantowego E < V) \\

Podstawiając za wartość energii elektronu 4.5 eV, a za wartość energii potencjalnej bariery 5 eV otrzymamy prawdopodobieństwo przejścia elektronu przez barierę o rozmiarze 5nm wynoszące około $ 2 \cdot 10^{-16} $. To bardzo mała wartość, jednak jeśli zmniejszymy rozmiar bariery do 0.5 nm (500 pm) – prawdopodobieństwo to przekroczy już 0.25 – wartość zdecydowanie nieakceptowalna z punktu widzenia praktycznych zastosowań. Jest więc wyraźnie widoczne, że już w niedalekiej przyszłości przestanie być możliwe zwiększanie mocy obliczeniowej komputerów w sposób, w jaki odbywa się to obecnie. Pojawia się zatem potrzeba zbadania alternatywnych metod, które pozwoliłyby ulepszać pracę urządzeń elektrycznych, bez nadwątlania rzetelności zwracanych przez nie rezultatów.


\section{Algebra liniowa jako język teorii kwantów}

Fizykę kwantową można ogólnie sformułować w dwóch, ekwiwalentnych postaciach matematycznych - postaci macierzowej\footnote{Heisenberg 1925} albo w postaci różniczkowej\footnote{Schrodinger 1926}. Dla celów niniejszej pracy wygodniejsza będzie pierwsza z nich. Celem niniejszego podrozdziału jest wprowadzenie pojęć niezbędnych do zrozumienia działania algorytmu \textit{Quantum Annealing} (przy czym nie jest to niezbędne do \textit{używania} tego algorytmu, szczególnie przy prostych problemach). 

\subsection{Przestrzeń liniowa oraz przestrzeń unitarna}

Przestrzeń liniowa $\mathbb{V}$ jest do zbiór obiektów (\textbf{a}, \textbf{b}, \textbf{c}, ...) dla których zdefiniowano operacje dodawania (\textbf{a} + \textbf{b}) oraz mnożenia przez tzw. skalar (\textit{a}, \textit{b}, \textit{c}, ...), przy czym: \newline
\begin{itemize}
    \item wyniki tych działań również są elementami tej przestrzeni, i.e. $\textbf{a} + \textbf{b} \in \mathbb{V}$ oraz $\textit{a}\textbf{a} \in \mathbb{V}$ (przestrzeń jest zamknięta ze względu na te działania) \newline
    \item mnożenie elementów przestrzeni przez skalar jest:
    \begin{itemize}
        \item rozdzielne względem dodawania elementów: \newline
         $\textit{a}(\textbf{a} + \textbf{b}) = \textit{a}\textbf{a} + \textit{a}\textbf{b}$
         \item rozdzielne względem dodawania skalarów: \newline
         $(\textit{a} + \textit{b})\textbf{a} = \textit{a}\textbf{a} + \textit{b}\textbf{a}$
         \item łączne: \newline
         $\textit{a}(\textit{b}\textbf{a}) = \textit{a}\textit{b}\textbf{a}$ \newline
         
    \end{itemize}
    \item dodawanie elementów przestrzeni jest:
    \begin{itemize}
        \item przemienne: \newline
        \textbf{a} + \textbf{b} = \textbf{b} + \textbf{a}
        \item łączne: \newline
        \textbf{a} + (\textbf{b} + \textbf{c}) = (\textbf{a} + \textbf{b}) + \textbf{c} \newline
    \end{itemize}
    \item istnieje element $\textbf{0}\in\mathbb{V}$ taki, że \newline $\forall(\textbf{a}\in\mathbb{V})$  $\textbf{a} + \textbf{0} = \textbf{a}$ \newline
    \item dla każdego elementu \textbf{a} istnieje element przeciwny \textbf{-a} taki, że \newline $\forall(\textbf{a}\in\mathbb{V})$  $\textbf{a} + \textbf{-a} = \textbf{0}$ \newline
\end{itemize}

Pewnym szczególnie istotnym z punktu widzenia niniejszej pracy zbiorem przestrzeni liniowych, będą takie, dla których zdefiniowano dodatkowo \textit{iloczyn skalarny}.\newline

Iloczyn skalarny definiowany jest jako funkcja $f: \mathbb{V}\times\mathbb{V}\to\mathbb{C}$, która dowolnej parze wektorów $\textbf{a}, \textbf{b} \in \mathbb{V}$ przyporządkowuje liczbę $f(\textbf{a}, \textbf{b})\equiv\langle\textbf{a},\textbf{b}\rangle\in\mathbb{C}$. Funkcja ta musi spełniać następujące warunki:
\begin{itemize}
    \item $\langle\textbf{a},\textbf{b}\rangle = \langle\textbf{b},\textbf{a}\rangle ^\ast$  (warunek sprzężonej symetrii)
    \item $\langle\textbf{a},\textbf{a}\rangle \geq 0$ (równość zachodzi wtedy i tylko wtedy, gdy $\textbf{a} = \textbf{0}$)
    \item $\langle\textbf{a},k\textbf{b} + l\textbf{c}\rangle = k\langle\textbf{a},\textbf{b}\rangle + l\langle\textbf{a},\textbf{c}\rangle$
\end{itemize}

Takie przestrzenie liniowe z iloczynem skalarnym nazywane są \textit{przestrzeniami unitarnymi}. Obliczenia kwantowe przeprowadza się na tzw. przestrzeniach Hilberta. Na potrzeby niniejszej pracy przez przestrzeń Hilberta będzie rozumiana skończono-wymiarowa przestrzeń unitarna, określona nad ciałem liczb zespolonych. Elementy takiej przestrzeni nazywane są \textit{wektorami}.
Wyróżniamy ponadto tzw. przestrzenie dualne, będące izomorficzne z przestrzeniami Hilberta. Pomimo tego ich wprowadzenie jest, z punktu widzenia matematycznego formalizmu, niezbędne do umieszczonego w następnej sekcji opisu notacji Diraca. 

\subsection{Notacja Diraca i bazy ortonormalne}

Niech $\mathcal{H}$ oznacza n-wymiarową przestrzeń Hilberta, zaś $\mathcal{H}^\ast$ - n-wymiarową przestrzeń dualną. \textit{Ketem} określa się wektor kolumnowy $\ket{V}\in\mathcal{H}$. Z kolei przez \textit{bra} rozumiany jest wektor \textit{wierszowy} $\bra{W}\in\mathcal{H}^\ast$. W klasycznej notacji \textit{bra} i \textit{ket} oznaczają listy liczb rozmiaru n, ułożone odpowiednio w pojedynczy wiersz i kolumnę (gwiazdka oznacza sprzężenie zespolone).
~\newline
~\newline
\begin{center}
    

$\ket{V} = \begin{bmatrix}
           w_{1} \\
           w_{2} \\
           \vdots \\
           w_{n}
           \end{bmatrix}, \;\;\; \bra{W} = \begin{bmatrix}
           v_{1}^\ast \; 
           v_{2}^\ast \;
           \hdots \;
           v_{n}^\ast
           \end{bmatrix}$
\end{center}

Wektor \textit{bra} z przestrzeni $\mathcal{H}^\ast$ otrzymujemy poprzez transpozycję oraz sprzężenie zespolone wektora \textit{ket} z przestrzeni $\mathcal{H}$.\newline

Iloczyn skalarny dla przestrzeni Hilberta jest zdefiniowany, jako:
\begin{center}
$\bra{W}\ket{V} = \sum_{i=1}^{n} w^\ast \cdot v$
\end{center}
Analogicznie definiujemy iloczyn skalarny w przestrzeni dualnej\footnote{Istotnie, także dalsze stwierdzenia na temat przestrzeni Hilberta będą prawdziwe dla przestrzeni dualnej, ze względu na izomorfizm tych przestrzeni}.\newline

Widać więc, że dla każdego $\ket{A}\in\mathcal{H}$, długość wektora $|\ket{A}|$ równa jest:
\begin{center}
    $|\ket{A}| = \sqrt{\braket{A}}$
\end{center}

Bazą ortonormalną w n-wymiarowej przestrzeni Hilberta nazywamy każdy zbiór elementów tej przestrzeni $\{\ket{1}, \ket{2}, \hdots, \ket{n}\}$, dla których:
\begin{center}
    $\bra{i}\ket{j} = \delta_{ij}$
\end{center}
gdzie $\delta_{ij}$ jest \textit{deltą Kroneckera}\footnote{Dla wartości i, j \textit{delta Kroneckera} przyjmuje wartość 1, gdy $i = j$, oraz 0 - w przeciwnym przypadku}.

Każdy wektor w przestrzeni Hilberta można zapisać jako kombinację liniową elementów bazy ortonormalnej tej przestrzeni:
\begin{center}
    $\forall_{\ket{V}\in\mathcal{H}}\ket{V} = \sum_{i=1}^{n}k_{i}\ket{i}$
\end{center}
gdzie $k_{i}$ jest liczbą (w ogólności zespoloną).

\subsection{Operatory liniowe}

\section{Fundamentalne zjawiska w informatyce kwantowej}

U podstaw informatyki kwantowej leżą dwa, wymykające się intuicji życia codziennego zjawiska - superpozycji i splątania. Pierwsze z nich pozwala na wykładnicze przyspieszenie obliczeń przeprowadzanych na komputerach kwantowych, drugie zaś jest niezbędne do konstrukcji maszyn operujących na układach tzw. \textit{qubitów} (kwantowych bitów). Przed omówieniem tych zjawisk konieczne jest jednak wyjaśnienie najpierw pojęcia samego \textit{qubitu} oraz jego reprezentacji matematycznej.

\subsection{Stan kwantowy jako wektor w przestrzeni Hilberta}

Podstawową jednostką danych w klasycznej informatyce jest bit. Fizyczną realizacją bitu jest każdy układ, który może w danej chwili znajdować się w jednym z dwóch stanów. W konwencjonalnych komputerach rolę bitów pełnią tranzystory - mogą one znajdować się w stanie niskiego (kodowanego jako 0) lub wysokiego (kodowanego jako 1) napięcia elektrycznego.

W komputerach kwantowych analogiczną do bitu rolę pełnią \textit{kwantowe bity} (\textit{quantum bits}) - w skrócie \textit{qubity}. Ich fizyczną realizacją również są układy mogące przyjmować jeden z dwóch różnych stanów, nazywanych stanami bazowymi. Przykładem tego jest spin elektronu, który przyjmuje jedną z dwóch wartości (spin "w górę" ~albo spin "w dół"\footnote{tutaj napisz coś o tym spinie}). To co je odróżnia, a co jednocześnie przesądza o ich niezwykłej użyteczności w teorii informacji, jest fakt, że stan \textit{qubitu} może być także dowolną \textit{kombinacją liniową} jego stanów bazowych.

W języku algebry liniowej oznacza to, że kwantowy bit można przedstawić, jako wektor w 2-wymiarowej przestrzeni Hilberta, a stany bazowe są wektorami bazy ortonormalnej tejże przestrzeni.
\begin{center}
    $\mathcal{H} \ni \ket{\textit{qubit}} = \alpha\ket{0} + \beta\ket{1}$
\end{center}
gdzie $\alpha, \beta \in \mathbb{Z}$.

Powyższe równianie interpretuje się w następujący sposób: \textbf{Stan kwantowego bitu}, wyrażony jako wektor w (2-wymiarowej) przestrzeni Hilberta, jest kombinacją liniową \textbf{stanu bazowego} $\ket{1}$ oraz \textbf{stanu bazowego} $\ket{2}$.

\subsection{Superpozycja stanów kwantowych}

Opisane powyżej algebraiczne wyrażenie stanu \textit{qubitu} jest czymś więcej, niż tylko zapisem matematycznym. Istotnie, fizyczna realizacja kwantowego bitu może się znajdować w stanie \textit{superpozycji kwantowej} - być jednocześnie w obu stanach bazowych. Elektron może mieć spin \textit{jednocześnie} "w górę" i "w dół". Jednakowoż, w momencie pomiaru spinu elektronu, eksperyment wskażę dokładnie jedną z tych dwóch wartości - nic pomiędzy. Wykonawszy ten eksperyment na dużej ilości elektronów otrzyma się pewną ilość wyników "góra" i pewną ilość "dół". W szczególności, dla stanu elektronu opisanego kombinacją liniową z poprzedniego podrozdziału, $|\alpha|^2$ elektronów będzie miało spin "w górę", a $|\beta|^2$ elektronów - "w dół". Współczynniki $\alpha$ oraz $\beta$ przy stanach bazowych interpretuje się jako amplitudy prawdopodobieństwa, natomiast kwadraty ich modułów - jako prawdopodobieństwa, że w momencie pomiaru stan \textit{qubitu} dozna \textit{kolapsu} do danego stanu bazowego.

Kolaps zachodzi zawsze - nie można wprost zaobserwować superpozycji stanów kwantowych. Powstaje zatem pytanie, czy zjawisko to rzeczywiście ma miejsce. Być może pozorna losowość otrzymywanych wyników eksperymentów jest jedynie pozorna, wynika z chaotyczności generowanej przez \textit{zmienne ukryte} (jak to jest na przykład w przypadku rzutu monetą - pozorna losowość wyniku wiąże się ze zmiennymi, takimi jak opór powietrza, siła przyłożona do monety w momencie rzutu itp.). 

\subsection{Zjawisko kwantowego splątania}

\section{Procedura \textit{Quantum Annealing} na maszynie \textit{D-Wave}}

\subsection{Model Isinga}

\subsection{Problemy klasy QUBO}

\subsection{Zagadnienie \textit{MIS} w ujęciu \textit{QUBO}}

\subsection{Implementacja algorytmu za pomocą solvera HSSv2}




\chapter{Zastosowanie algorytmu \textit{QA} do problemu dywersyfikacji portfela}

\section{Instalacja i konfiguracja biblioteki \textit{dwave-ocean-sdk}}
\subsection{Instalacja}

\subsection{Konfiguracja i wybór solvera}

\section{Architektura projektu}

\subsection{Załadowanie danych}

\subsection{Przygotowanie danych wejściowych}

\subsection{Uruchomienie procedury i wizualizacja wyników}

\section{Ostateczne wyniki procedury}

\chapter{Analiza rezultatów}
\label{sec:nast}

Bieżący rozdział (czyli rozdział~\ref{sec:nast}) zawiera odwołanie do samego siebie. Stosujmy odwołania automatyczne.

\clearpage

\chapter{Podsumowanie}

\clearpage
\addcontentsline{toc}{chapter}{Bibliografia}
\begin{thebibliography}{99}
\setlength{\itemsep}{0pt}%
\bibitem[Shankar R.(2006)]{shankar2006} Shankar R.(2006), Mechanika Kwantowa, Wydawnictwo Naukowe PWN SA, s.~19--29
\bibitem[Wakker i Zhank(1999)]{wakker1999} Wakker P., H.~Zank (1999), A Unified Derivation (...), \textit{Journal of  Mathematical Economics}, 32, s.~1-19
\bibitem[Drummond et al.(1997)]{drummond1997} Drummond M.F., B.J.~O'Brien, G.L.~Stoddart, G.W.~Torrance (1997), Methods for Economic Evaluation (...), Oxford University Press
\end{thebibliography}

\clearpage
\addcontentsline{toc}{chapter}{Spis rysunków}
\listoffigures

\clearpage
\listoftables
\addcontentsline{toc}{chapter}{Spis tabel}

\appendix
\chapter*{Kody źródłowe}
\addcontentsline{toc}{chapter}{Kody źródłowe}

\section*{Analiza 1}
\begin{verbatim}
x <- 1:10
y <- x + rnorm(10)
summary(lm(y ~ x))
\end{verbatim}

\section*{Analiza 2}
\begin{verbatim}
x <- rnorm(10000)
plot(density(x))
\end{verbatim}

\clearpage

\chapter*{Streszczenie}
\addcontentsline{toc}{chapter}{Streszczenie}

Tutaj piszemy streszczenie. Między 200 a 350 słów. Nie jest omówieniem struktury pracy. Zawiera cel, metodę, dane, wyniki, wnioski. Nie ma odwołań źródłowych, list, wykresów, tabel. Ma być zrozumiałe dla osoby nieczytającej pracy.

\end{document}

